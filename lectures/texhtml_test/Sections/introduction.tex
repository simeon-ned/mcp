Various applications of mechanical engineering and robotics involve periodic motion of the end-effectors, links and payloads. These include legged robots, lower-limb exoskeletons, mechanisms providing cyclic motion, and many others. One of the key aspects in control over such systems is their energy efficiency. This calls for the use of lightweight actuators with high power density and efficient control schemes for them. 

Twisted string actuators (TSAs) are rotary-to-translational cable-driven actuators, in which torsional twisting of a bundle of strings or cables results in their contraction. Key advantages of TSAs include light weight, high efficiency and power density, and compliance. These actuators are used in various applications including multi-legged robots \cite{suzuki2005toward}, lower-limb exoskeletons \cite{kornbluh2018twisted} and mobile robots \cite{sabelhaus2015system}, all of which involve periodic motion of the payload. However, the dynamical properties of TSAs have not been fully explored in these studies, with the authors mainly employing the actuators for low-speed control. However, these applications involved comparatively low payload forces \cite{nedelchev2020accurate}. When operating with significant payload during high-frequency motion, a more efficient control approach might be required.

Conventionally, TSAs are controlled in a uni-directional fashion when the motor twists the strings from their initial state all the way until desired contraction level is reached, and then back. Thus, if a fast periodical motion of the payload is required, the motor needs to be stopped and reversed 4 times during each period, around both zero and fully twisted positions, which may cause power losses since the energy generated by motor driver should be sufficient to overcome the kinetic energy of spinning rotor inertia when stopping. However, one of the unique features of the TSAs is their symmetrical behaviour: the strings contract identically whether they are twisted clockwise or counterclockwise. With this in mind, one can design a control system in such a way that the motor twists the strings in both directions, which will eliminate the need to reverse the motor in its zero configuration. An additional benefit is that motor's shaft itself, despite its comparatively low inertia, can propel the payload by twisting the strings on its own without drawing any power from the supply. As a consequence, a TSA-driven system with proposed controller can reach higher string contraction values with constraints on motor torque.

%In this paper, we implement an energy-preserving control in application to twisted string actuators. 
We have observed during the experiments that when the loaded twisted strings were released freely from their contracted state, they  twisted in the opposite direction on their own after being completely untwisted, carried on by the motor inertia. As the result, the strings contracted again, sometimes working against significant forces that acted on the payload, with no effort required from the actuator. This held true even for comparatively small values of rotor inertia.
Thus, knowing some intrinsic properties of a TSA-powered setup such as its natural frequency, one can leverage the tendency of TSA to oscillate freely in order to design an efficient controller which will be able to induce periodical motion of the payload with minimal motor power required. This controller can maintain the mechanical energy of the system at a desired, constant level by compensating for the power losses within the system, effectively turning the setup into a frictionless nonlinear pendulum. In this work, we have implemented an energy-preserving control law on a practical linear TSA setup, and designed it so that it supported undamped oscillatory response of a 2-kg payload with desired magnitude and natural frequency while  requiring a maximum of 6 mNm torque from the actuator. 

To the best of our knowledge, this paper marks the first occasion when free oscillations in TSA systems were studied, modeled, and experimentally investigated. In addition, this work presents novel experimental results on the application of energy-preserving control to TSA-based robotic systems. Such systems and controllers can be used in the limbs of bipedal and quadruped robots to ensure energy-efficient locomotion, even in the presence of severe actuator limitations.


