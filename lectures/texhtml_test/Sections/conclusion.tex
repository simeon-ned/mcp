In this paper, we investigated free oscillatory response of linear twisted string actuators. We have outlined a mathematical model based on energy observations and TSA dynamics that could predict the oscillations with satisfactory accuracy and described energy conversion within TSA. 

One important consequence of understanding the nature of oscillatory responses in twisted string actuators is the ability to design TSA-based robotic systems that could efficiently leverage the advantages of this oscillatory process for the purposes of control. As one example, with appropriate controller selection one can deliberately induce undamped oscillatory response in TSA by preserving mechanical energy within a system at a desired level. We have implemented one such technique (energy-preserving control) and validated it in a series of hardware experiments. The tests have demonstrated that it was possible to generate undamped oscillatory response in a TSA with a payload of 2 kg while exerting a maximum of 6 mNm torque. This suggests that energy preserving control can be used in combination with TSA to ensure power-efficient locomotion, and this control strategy can also be used in applications that pose high requirements to power efficiency and require periodic movements of a certain magnitude.

In the simplest TSA module like the one studied in this paper, we can only control the magnitude of the oscillations and not their frequency. However, if one introduces an element with variable stiffness, it should become possible to control both frequency and magnitude, thus supporting a more diverse range of periodic motions. This can be beneficial for applications like legged robots that may require different gaits depending on a situation, or for robots whose dynamical parameters can vary significantly during periodic movements.

In the nearest future, we are planning to investigate oscillations and extend energy preserving control to rotational TSA-based systems. In addition, we are planning to design a 2-DOF robotic legged that employs TSAs, induce oscillatory response and investigate the effects of contact with environment (hybrid dynamics) on system behavior. When developing TSA-based robotic limbs, one can design their movement trajectories in such a way that they are as close to oscillations as possible. Once designed, the controller will stabilize the resulting orbits.