Many applications require robotic end-effectors and mechanisms to move along periodic trajectories of given amplitude and frequency.
If motion parameters are known in advance, it might be beneficial to design the mechanism in such a way that its natural frequency is close to that of the desired trajectory so that controller needs to provide minimal effort to generate sustained oscillations. 
In this paper, we investigate natural nonlinear oscillatory behavior in twisted  string  actuators (TSA), describing its mathematical model and providing experimental verification of this phenomenon based on observations of dynamics and energy.
We also design an energy-preserving controller that is capable of generating undamped oscillations of desired magnitude even under severe constraints on actuator torque.
Experimental study has demonstrated that it was possible to induce undamped oscillatory response of TSA with a 2-kg payload while applying a maximum of 6 mNm motor torque, which can be used in robotic applications that require periodic motions and high controller efficiency, like legged robots.